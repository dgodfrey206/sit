\section{命令介绍}
sit使用了boost中的program\_options模块来格式化处理命令行参数。除了非默认参数外,其余参数在使用时均需要满足下列格式之一:
\begin{lstlisting}[language=sh]
--<option_name>=<option_value>
-<option_abbr>=<option_value>
\end{lstlisting}

其中等号可以用空格代替。默认参数可以省略掉等号前面的部分。

下文提到的{\YaHeiMono <commit-id>}都代表此处接受某个commit的SHA1值的足够长的前缀或版本指针的名字(如HEAD,index,work,master)。
\subsection{add}
\subsubsection* {Introduction}
把文件内容加入到index
\subsubsection*{Usage}
\begin{lstlisting}[language=sh]
sit add <path_1> [<path_2> ...]
\end{lstlisting}
\subsubsection*{Options}
\begin{description}
	\item[\YaHeiMono <path>(默认参数)] 要加入index的文件(或文件夹,如果是文件夹将递归加入该文件夹下所有的子文件(夹))
\end{description}

\subsection{checkout}
\subsubsection*{Introduction}
将代码库/index中的文件解压并复制到当前的工作区中。
\subsubsection*{Usage}
checkout命令有两种基本形式:
\begin{lstlisting}[language=sh]
sit chekcout [--commit=<commit-id>]
\end{lstlisting}
\begin{lstlisting}[language=sh]
sit checkout [--commit=<commit-id>] --path=<path_1> [--path=<path_2> ...]
\end{lstlisting}
\subsubsection*{Options}
\begin{description}
	\item[\YaHeiMono commit] 指定复制到工作区的文件来源,接受的值为任意一个commit的ID的前缀(可以自动补全)或者是版本指针(目前支持的有HEAD,index,master),默认值为index。
	\item[\YaHeiMono path(默认参数)] 指定需要复制到工作区的文件(夹),如果没有指定任何一个文件,那么checkout将复制该commit的所有文件到工作区,并且将HEAD指针指向checkout的commit(如果commit参数为index则不改变HEAD指针)。
\end{description}

\subsection{config}
\subsubsection*{Introduction}
修改该仓库的配置,目前支持的仓库配置有用户名和用户的Email地址(user.name和user.email)。
\subsubsection*{Usage}
\begin{lstlisting}[language=sh]
sit config <key> <value>
\end{lstlisting}
\subsubsection*{Options}
\begin{description}
	\item[\YaHeiMono <key>] 修改配置的字段名
	\item[\YaHeiMono <value>] 新的配置值
\end{description}

\subsection{commit}
\subsubsection*{Introduction}
将当前``.sit/index''文件中记录的文件信息写入仓库。
\subsubsection*{Usage}
\begin{lstlisting}[language=sh]
sit commit  [--amend] [--all | -a] [--message | -m COMMIT_MSG]
\end{lstlisting}
\subsubsection*{Options}
\begin{description}
	\item[\YaHeiMono all,a] 等价于在执行commit操作之前先执行一次`add <REPO\_ROOT>',其中REPO\_ROOT表示仓库的绝对路径。
	\item[\YaHeiMono amend] 表示新的commit将取代原来的HEAD所指向的commit在版本链中的位置。\\\textbf{注意:如果没有amend参数,那么commit只有当HEAD指针与master指针相等的时候才会执行(为了保证版本链的唯一)。}
	\item[\YaHeiMono message,m] 以参数的形式提供用于描述该commit改动的信息。
\end{description}

\subsection{diff}
\subsubsection*{Introduction}
比较两个commit的文件信息(或两个不同版本的同一文件(夹))的差异。
\subsubsection*{Usage}
\begin{lstlisting}[language=sh]
sit diff [--base-id=<commit-id>] [--target-id=<commit-id>] [--path=<path> ...]
\end{lstlisting}
\subsubsection*{Options}
\begin{description}
	\item[\YaHeiMono base-id] 作为基准的commit的ID,默认值为index。
	\item[\YaHeiMono target-id] 比较差异的commit的ID,默认值为work(即当前的工作区)。
	\item[\YaHeiMono path(默认参数)] 用于指定需要比较的文件(夹),如果没有指定任何一个文件,那么将比较两个版本的index以及每一个有修改的文件。
\end{description}

\subsection{gc}
\subsubsection*{Introduction}
删除那些``不再需要''的objects,释放它们占用的硬盘空间。
\subsubsection*{Usage}
\begin{lstlisting}[language=sh]
sit gc
\end{lstlisting}

\subsection{init}
\subsubsection*{Introduction}
在当前的工作目录下建立一个sit的版本仓库,如果已经存在了一个sit的版本仓库,那么将把这个版本仓库删除后再创建
(即初始化)。
\subsubsection*{Usage}
\begin{lstlisting}[language=sh]
sit init
\end{lstlisting}

\subsection{log}
\subsubsection*{Introduction}
以当前HEAD指针指向的commit为最后的commit,输出该commit以及它的所有父版本的信息(作者,提交时间,提交信息)。
\subsubsection*{Usage}
\begin{lstlisting}[language=sh]
sit log
\end{lstlisting}

\subsection{status}
\subsubsection*{Introduction}
将当前的工作区的index和HEAD指针所指向的commit的index进行比较,分别输出新增的文件、被修改的文件、被删除的文件的列表。
\subsubsection*{Usage}
\begin{lstlisting}[language=sh]
sit status
\end{lstlisting}

\subsection{reset}
\subsubsection*{Introduction}
将仓库中某个commit中保存的文件信息恢复到当前的index中。
\subsubsection*{Usage}
reset命令有两种形式:
\begin{lstlisting}[language=sh]
sit reset [--hard] [--commit=<commit-id>]
\end{lstlisting}
\begin{lstlisting}[language=sh]
sit reset  [--commit=<commit-id>] --path=<path_1> [--path=<path_2> ...]
\end{lstlisting}

第一种形式会将当前分支的版本指针移动到指定的commit上。(移动版本链的链表尾,操作后尾后的commit还会存在仓库中,此时执行gc命令将把这些commit相关的且没有被其他文件引用的objects清理掉。)

第二种形式必须指定最少要恢复的文件,否则sit会将其解释为第一种形式。
\subsubsection*{Options}
\begin{description}
	\item[\YaHeiMono hard] 如果给出``--hard''参数,那么reset命令在恢复index中的文件信息的同时,会从仓库中复制相应的文件到工作区。如果没有则reset命令只会对index进行修改。
	\item[\YaHeiMono commit] 用于指定要恢复到index的文件信息来自于哪一个commit,默认值为HEAD。
	\item[\YaHeiMono path(默认参数)] 用于指定恢复哪些文件的信息。\textbf{注意:path参数和hard参数不能同时存在。如果必要的话,可以先后执行reset和checkout两个命令来满足要求。}
\end{description}

\subsection{rm}
\subsubsection*{Introduction}
把文件(夹)的记录从index中移除,不会修改工作区的文件(夹)。
\subsubsection*{Usage}
\begin{lstlisting}[language=sh]
sit rm --path=<path_1> [--path=<path_2> ...]
\end{lstlisting}
\subsubsection*{Options}
\begin{description}
	\item[\YaHeiMono path(默认参数)] 表示要从index中移除的文件(夹)。
\end{description}